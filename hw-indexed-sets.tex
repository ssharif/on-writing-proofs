%%% Local Variables:
%%% TeX-master: "Proofs"
%%% End:
\probsec{~\ref{sec:indexed-sets}}
\begin{enumerate}
    \item For $i \in \N$, let $A_i = \{-i, -i+1, \dots, i-1, i\}$.
  \begin{enumerate}
      \item Write $A_i$ in set-builder notation.
      \item What is $A_1 \cup A_2 \cup A_3$?
      \item What is $A_1 \cup A_2 \cup \cdots \cup A_{n}$?
      \item What is $\displaystyle\bigcup_{i \in \N} A_i$?
      \item What is $\displaystyle\bigcap_{i\in\N} A_i$?
  \end{enumerate}

    \item Let $I_n = (\frac{1}{n}, 1]$. (This is interval notation.) What is $\displaystyle\bigcup_{n \in \N} I_n$? You do not need to prove your answer.

    \item For $n \in \N$, let $A_n$ be the interval $(-\frac{1}{n}, \frac{1}{n})$. What are $\cup A_n$ and $\cap A_n$?\sidenote{The omission of the index means over all possible $n$. This is analogous to writing, for example, $\sum a_n$ instead of $\displaystyle\sum_{n=1}^{\infty} a_n$.} You do not need to prove your answer.

    \item Let $A_y \subset \R^2$ where $y \in \R$ be given by $A_y = \{(x, x+y): x \in \R\}$. You do not need to prove your answers.
  \begin{enumerate}
      \item Graph $A_1 \cup A_2 \cup A_3$.
      \item Graph $\displaystyle\bigcup_{y \in \Z} A_y$.
      \item Graph $\displaystyle\bigcup_{y \in \N} A_y$.
      \item Graph $\displaystyle\bigcup_{y \in \R} A_y$.
      \item Graph $\displaystyle\bigcup_{y \geq 0} A_y$.
  \end{enumerate}

    \item Let $B_x \subset \R^2$ where $x \in \R$ be given by $B_x = \{(x, x+y): y \in \R\}$. You do not need to prove your answers.
  \begin{enumerate}
      \item Graph $B_1 \cup B_2 \cup B_3$.
      \item Graph $\displaystyle\bigcup_{x \in \Z} B_x$.
      \item Graph $\displaystyle\bigcup_{x\in \N} B_x$.
      \item Graph $\displaystyle\bigcup_{x \in \R} B_x$.
      \item Graph $\displaystyle\bigcup_{x \geq 0} B_x$.
  \end{enumerate}

    \item For $n \in \Z$, let $A_n \subset \Z$ be the set of multiples of $n$. What are $\cup A_n$ and $\cap A_n$?

    \item Let $A_\alpha = \{(x,y): |y| \leq x^2/\alpha\}$ where $\alpha \in \R$ and $\alpha > 0$. Graph $\cap A_\alpha$ and $\cup A_\alpha$.

    \item Suppose $A_i \subset \cap A_j$ for every $i$. What can you conclude?

    \item Suppose $\cup A_j \subset A_i$ for every $i$. What can you conclude?

    \item For $(r, n) \in \Q \times \N$, let $A_{(r,n)}$ be the interval $(r - \frac{1}{n}, r + \frac{1}{n})$. Compute the following. You do not need to prove your answers.
  \begin{enumerate}
      \item $\displaystyle\bigcap_{r \in \Q} \bigcup_{n \in \N} A_{(r,n)}$
      \item $\displaystyle\bigcup_{r \in \Q} \bigcap_{n \in \N} A_{(r,n)}$
      \item $\displaystyle\bigcup_{n \in \N} \bigcap_{r \in \Q} A_{(r,n)}$
      \item $\displaystyle\bigcap_{n \in \N} \bigcup_{r \in \Q} A_{(r,n)}$
  \end{enumerate}

\end{enumerate}
