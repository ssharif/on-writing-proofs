\probsec{~\ref{sec:conjunctions}}
\begin{enumerate}
    \item You meet Allen, Burroughs, and Corso.
  \begin{dialogue}
    \speak{Allen} Burroughs and Corso are  \knights.
    \speak{Burroughs} Either I or Allen is a \knave.
  \end{dialogue}
  What can you conclude?

    \item You meet Anand, Botvinnik, and Capablanca.
  \begin{dialogue}
    \speak{Anand} Either Botvinnik is a \knave or Capablanca is a \knight.
    \speak{Botvinnik} Capablanca and I are both \knaves.
  \end{dialogue}
  What can you conclude?

    \item You encounter Asia, Brussels, and Cuba.
  \begin{dialogue}
    \speak{Asia} Cuba and I are both \knights.
    \speak{Brussels} Either Asia or Cuba is a \knave.
    \speak{Cuba} Asia and Brussels are both \knaves.
  \end{dialogue}
  What can you conclude?

  \item You meet Arkana, Bullard, Corinthian, and Robin.
  \begin{dialogue}
    \speak{Arkana} Either Bullard is a \knight or Corinthian is a \knave.
    \speak{Bullard} Either Arkana or Robin is a \knave.
    \speak{Corinthian} At least one of the other three is a \knave.
    \speak{Robin} The sky is green.
  \end{dialogue}
  You know that there is at least one \knight and at least one \knave amongst the three. What can you conclude?

  \item You encounter Al, Bonnie, Clyde, and Dillinger. One of them is a \knave who stole a very valuable mathematical manuscript. Determine from their statements who is the thief.
  \begin{dialogue}
    \speak{Al} Dillinger is the thief.
    \speak{Bonnie} At least one of us is a \knight.
    \speak{Clyde} At least one of us is a \knave.
    \speak{Dillinger} Both Bonnie and Clyde are \knights.
  \end{dialogue}

\end{enumerate}
