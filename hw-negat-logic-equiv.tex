%%% Local Variables:
%%% TeX-master: "Proofs"
%%% End:
\probsec{~\ref{sec:negat-logic-equiv}}

\noindent For these problems, you do not need to prove your answers unless explicitly stated.

\begin{enumerate}
    \item The statement ``Neither P nor Q'' is logically equivalent to what statement that we've already seen?

    \item Rewrite each of the following statements into a logically equivalent statement.\sidenote{These should be grammatically inequivalent.}
  % These examples necessitate the generalities below.
  \begin{enumerate}
      \item She is either not blonde or not a mathematician.
      \item If it is raining, then the grass is wet.
      \item Whenever the sun is visible, it is daytime.
      \item The number $x$ is not an odd perfect square.
  \end{enumerate}

    \item Negate each of the statements in the previous problem.

    \item Rewrite each of the following statements into a logically equivalent statement.
  \begin{enumerate}
      \item The jewel is either blue or shiny, but not both.
      \item If $S$ does not have 4 sides, then it is not a square.
      \item The number $x$ is prime, and it either ends in a 1 or is
    bigger than 1000.
  \end{enumerate}

    \item Negate each of the statements in the previous problem.

    \item Show that conditionals are not logically equivalent to their converses by using a truth table.

    \item Write out the truth table for ``If not Q then not P''.

    \item Prove part (b) of DeMorgan's Laws in two different ways: with a proof written in paragraph form, and with a truth table.

  %   \item 
  % \begin{enumerate}
  %     \item Prove Theorem~\ref{thm:negation-conditional} in two ways: with a paragraph, and with a truth table.
  %     \item Find an example for P and Q that makes both ``Not `if P then Q'~''\sidenote{This looks clunky; if you prefer, ``The statement `If P then Q' is false'' means the same thing.} and ``Not P and Q'' true.
  %     \item Find an example for P and Q that makes both statements false.
  % \end{enumerate}

    \item The statement ``If P then Q'' is equivalent to a statement of the form ``\underline{\hspace{.5in}} or \underline{\hspace{.5in}}''.
  \begin{enumerate}
      \item Fill in the blanks, and prove your answer with either a proof in paragraph form or a truth table.
      \item Give an example of P and Q so that both statements above are true, and write down the corresponding statements.
      \item Now give an example where both statements are false.
  \end{enumerate}

    \item Prove that ``If P, then Q or R'' is equivalent to ``If P and not Q, then R'' without using a truth table.
\end{enumerate}
