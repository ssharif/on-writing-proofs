%%% Local Variables:
%%% TeX-master: "Proofs"
%%% End:
\probsec{~\ref{sec:negat-logic-equiv}}
\begin{enumerate}
    \item Write out the truth table for ``If not Q then not P''.

    \item Prove part (b) of DeMorgan's Laws.

    \item Consider the conditional ``If P then Q'' and its converse. In the following, you are asked to come up with an example P and Q which satisfies certain conditions. That means that you should choose a specific statement for P (like ``clowns are scary''\sidenote{True fact.}) and a specific statement for Q in such a way that the condition is satisfied.
  \begin{enumerate}
      \item Come up with an example of P and Q which makes both statements true; write down both statements.
      \item Same question, but now both statements should be false.
      \item Same question, but now ``If P then Q'' is true and the converse is false.
      \item Same question, but now ``If P then Q'' is false and the converse is true.
  \end{enumerate}

    \item 
  \begin{enumerate}
      \item Prove Theorem~\ref{thm:negation-conditional} in two ways: with a paragraph, and with a truth table.
      \item Find an example for P and Q that makes both ``not `if P then Q' ''\sidenote{This looks clunky; if you prefer, ``The statement `if P then Q' is false'' means the same thing.} and ``not P and Q'' true.
      \item Find an example for P and Q that makes both statements false.
  \end{enumerate}

    \item The statement ``If P then Q'' is equivalent to a statement of the form ``\underline{\hspace{.5in}} or \underline{\hspace{.5in}}''.
  \begin{enumerate}
      \item Fill in the blanks, and write down a truth table for the latter statement.
      \item Give an example of P and Q so that both statements are true, and write down the corresponding statements.
      \item Now give an example where both statements are false.
  \end{enumerate}

\end{enumerate}
