%%% Local Variables:
%%% TeX-master: "Proofs"
%%% End:
\probsec{~\ref{sec:conditionals}}
\begin{enumerate}
    \item You meet Asimov, Banks, and Clarke.
  \begin{dialogue}
    \speak{Asimov} If Banks is a \knight, then Clarke is a \knave.
    \speak{Banks} Both Clarke and I are \knaves.
  \end{dialogue}
  What can you conclude?

    \item You encounter Anthony, Beauvoir, and Chisholm.
  \begin{dialogue}
    \speak{Anthony} If I am a \knave, then Beauvoir is a \knight.
    \speak{Beauvoir} Both Chisholm and I are \knights.
    \speak{Chisholm} Either I or Beauvoir is a \knave.
  \end{dialogue}
  What can you conclude?

    \item You encounter Adams, Buchanan, and Carter.
  \begin{dialogue}
    \speak{Adams} If Buchanan is a \knight, then Carter is a \knave.
    \speak{Buchanan} Adams is a \knave.
    \speak{Carter} If Buchanan is a \knight, then I am a \knave.
  \end{dialogue}
  What can you conclude?

    \item You run into Avalanche, Blizzard, and Cyclone.
  \begin{dialogue}
    \speak{Avalanche} If Blizzard is a \knight, then either I or Cyclone is a \knave.
    \speak{Blizzard} If Avalanche is a \knave, then both I and Cyclone are \knights.
    \speak{Cyclone} Both Avalanche and Blizzard are \knaves.
  \end{dialogue}
  What can you conclude?

    \item Consider the conditional ``If P then Q'' and its converse. In the following, you are asked to come up with an example P and Q which satisfies certain conditions. That means that you should choose a specific statement for P (like ``clowns are scary''\sidenote{True fact.}) and a specific statement for Q in such a way that the condition is satisfied.
  \begin{enumerate}
      \item Come up with an example of P and Q which makes both statements true; write down both statements.
      \item Same question, but now both statements should be false.
      \item Same question, but now ``If P then Q'' is true and the converse is false.
      \item Same question, but now ``If P then Q'' is false and the converse is true.
  \end{enumerate}
\end{enumerate}
