%%% Local Variables:
%%% TeX-master: "Proofs"
%%% End:
\probsec{~\ref{sec:conditionals}}
\begin{enumerate}
    \item Determine if the following statements are true or false.
  \begin{enumerate}
      \item If this sentence is false, then it is true.
      \item If this sentence is true, then it is false.
  \end{enumerate}

    \item Your friend has 4 cards, each of which has a letter on one side and an integer on the other. Your friend then makes the statement, ``If the letter is a vowel, then the number on the other side is even.'' You want to check to see whether the statement is true or false. Suppose the 4 cards are lying on a table, and the sides visible to you read
  \[
  \framebox{A} \qquad \framebox{N} \qquad \framebox{4} \qquad \framebox{7}.
  \]
  Which cards do you turn over?\sidenote{Similar to the bouncer problem, you want to turn over the minimum number of cards.}

    \item You meet Asimov, Banks, and Clarke.
  \begin{dialogue}
    \speak{Asimov} If Banks is a \knight, then Clarke is a \knave.
    \speak{Banks} Both Clarke and I are \knaves.
  \end{dialogue}
  What can you conclude?

    \item You encounter Anthony, Beauvoir, and Chisholm.
  \begin{dialogue}
    \speak{Anthony} If I am a \knave, then Beauvoir is a \knight.
    \speak{Beauvoir} Both Chisholm and I are \knights.
    \speak{Chisholm} Either I or Beauvoir is a \knave.
  \end{dialogue}
  What can you conclude?

    \item You encounter Allende, Borges, and Cervantes.
  \begin{dialogue}
    \speak{Allende} If Borges is a \knight, then Cervantes is a \knave.
    \speak{Borges} Allende is a \knave.
    \speak{Cervantes} If Borges is a \knight, then I am a \knave.
  \end{dialogue}
  What can you conclude?

    \item You run into Avalanche, Blizzard, and Cyclone.
  \begin{dialogue}
    \speak{Avalanche} If Blizzard is a \knight, then either I or Cyclone is a \knave.
    \speak{Blizzard} If Avalanche is a \knave, then both I and Cyclone are \knights.
    \speak{Cyclone} Both Avalanche and Blizzard are \knaves.
  \end{dialogue}
  What can you conclude?

    \item Consider the conditional ``If P then Q'' and its converse. In the following, you are asked to come up with an example P and Q which satisfies certain conditions. That means that you should choose a specific statement for P (like ``clowns are scary''\sidenote{True fact.}) and a specific statement for Q in such a way that the condition is satisfied.
  \begin{enumerate}
      \item Come up with an example of P and Q which makes both statements (i.e.~the conditional and its converse) true; write down both statements.
      \item Same question, but now both statements should be false.
      \item Same question, but now ``If P then Q'' is true and the converse is false.
      \item Same question, but now ``If P then Q'' is false and the converse is true.
  \end{enumerate}
\end{enumerate}
