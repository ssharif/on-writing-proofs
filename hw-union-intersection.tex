%%% Local Variables:
%%% TeX-master: "Proofs"
%%% End:
\probsec{~\ref{sec:prod-union-inters}}
\begin{enumerate}
  \item Suppose $A$ is a set. Compute the following. You don't have to prove your answer.
\begin{enumerate}
    \item $A \times \emptyset$
    \item $A \cup \emptyset$
    \item $A \cap \emptyset$
    \item $A \cup (\emptyset \cap \emptyset)$
    \item $(A \cup \emptyset) \cap \emptyset$
\end{enumerate}

  \item Suppose $A = \{1, 2\}$ and $B = \{2, 3\}$. Write out the following sets in list form. You don't have to prove your answer.
\begin{enumerate}
    \item $((A \cup B) \cup B) \cup B$
    \item $A \times B$
    \item $A \times (B \cap A)$
    \item $(A \times B) \cap A$
    \item $A \cup (B \times A)$
    \item $(A \cup B) \times A$
    \item $A \cap (B \times A)$
    \item $(A \cap B) \times A$
\end{enumerate}

  \item Graph the following subsets of $\R^2$. You don't have to prove your answer.
\begin{enumerate}
    \item $\R \times \Z$
    \item $\Z \times \R$
    \item $[1,2] \times [-1,1]$\sidenote{This is interval notation. If you're confused, write out a few elements of the set.}
    \item $[1,2] \times \R$
    \item $\Z \times (-1,1)$
    \item $\{(x, y) \mid y - x^2 = 0\}$
    \item $\{(x, y) \mid 1 - x^2 - y^2 \geq 0\}$
    \item $\{(x, y) \mid x^2 - 1 = 0\}$
    \item $\{(x, x) \mid x \in \R\}$
    \item $\{(\cos t, \sin t) \mid t \in [0, \pi/2]\}$
    \item $\{(x, y) \mid x + y = -1\}$
    \item $\{(x, y) \mid x + y \in \Z\}$
\end{enumerate}

  \item Let $A$, $B$, and $C$ be as in \Cref{ex:filled-smile}. Graph the following.
\begin{enumerate}
    \item $(A \cup B) \cap C$.
    \item $(A \cap B) \cup C$.
    \item $(A \cup C) \cap B$.
    \item $(A \cap C) \cup B$.
\end{enumerate}

  \item Suppose $A$ and $B$ are finite sets with $\# A = m$ and $\# B = n$. What is $\# (A \times B)$?

  \item Suppose that $A$ and $B$ are finite sets.
\begin{enumerate}
    \item Show by example that $\# (A \cup B)$ does not necessarily equal $\# A + \# B$.
    \item Suppose $\# (A \cup B) = \# A + \# B$. What can you conclude?
\end{enumerate}

  \item Suppose $A$, $B$, and $C$ are sets.
\begin{enumerate}
    \item Prove that $(A \cup B) \cup C = A \cup (B \cup C)$.\marginnote{This property is of course called \emph{associativity}. When an operation is associative, we can remove the parentheses entirely; e.g.~$A \cup B \cup C$.}
    \item Prove that $(A \cap B) \cap C = A \cap (B \cap C)$.
\end{enumerate}

  \item If $A$, $B$, and $C$ are sets, prove the following.
\begin{enumerate}
    \item $(A \cap B) \cup (A \cap C) = A \cap (B \cup C)$.
    \item $(A \cup B) \cap (A \cup C) = A \cup (B \cap C)$.
\end{enumerate}


  \item Suppose $A$ and $B$ are sets. Prove that $A \cap B \subset A \cup B$.

  \item Suppose $A$ and $B$ are sets, and that $A \subset A \cap B$. What can you conclude?\sidenote{If you're thinking, ``How the heck should I know?'', then I highly recommend doing out several examples. That is, choose different pairs of sets $A$ and $B$, and test to see if $A \subset A \cap B$ holds or not. Then try to see if there is a pattern to the cases where equality does hold. Remember: you are expected to prove your answer unless explicitly stated otherwise.}

  \item Suppose $A$ and $B$ are sets, and that $A \cup B \subset A$. What can you conclude?

  \item Suppose that $A$ and $B$ are sets, and that $A \cap B = A \cup B$. What can you conclude?

\end{enumerate}
